\section{Configuring and Building Programs Using \cgal}


Ideally, configuring and compiling a demo/example/program amounts to

{\ccTexHtml{}{}
\begin{alltt}

cd \cgalrel/examples/Straight_skeleton_2
cmake -DCGAL_DIR=$HOME/\cgalrel . 
make

\end{alltt}
}

%$ % <- added to close non-correct "math" enviroment ("HOME"), DO NOT DELETE

In this ideal world, as for all shipped examples and demos of \cgal, the
required \texttt{CMakeLists.txt} is already provided. 

\cmake\ can also be used to configure and build user programs via such
\cmake-scripts. In this less ideal world, one has to provide the
\texttt{CMakeLists.txt} script either manually, or with the help of a
shell-script that is introduced below.

For a user program \texttt{executable.cpp}, the ideal world looks like this:

{\ccTexHtml{}{}
\begin{alltt}
cd /path/to/program 
cgal_create_CMakeLists -s executable
cmake -DCGAL_DIR=$HOME/\cgalrel . 
make
\end{alltt}
}

%$ % <- added to close non-correct "math" enviroment ("HOME"), DO NOT DELETE

In both examples we specify the \texttt{CGAL\_DIR}:
During configuration of the \cgal\ libraries a file named {\tt
CGALConfig.cmake} is generated in \cgal's root directory (in contrast
to \cgal's source directory that has been used for installation). This file
contains the definitions of several \cmake\ variable that summarize the
configuration of \cgal. In order to configure a program, you need
to indicate the location of that config file in the \cmake\ variable
\texttt{CGAL\_DIR} (as indicated in the example above). 
\texttt{CGAL\_DIR} can also be an environment variable. Setting
\texttt{CGAL\_DIR} makes particular sense if having multiple
out-of-source builds of \cgal\ as in Section~\ref{sec:cmake-out-of-source}.

If you have installed CGAL, \texttt{CGAL\_DIR} must afterwards be set to
\texttt{\$CMAKE\_INSTALLED\_PREFIX/lib/CGAL}. Note that \cgal\ is
recommended to be installed in release mode when using it to build programs.

\subsection{Creating a cmake script for a program using \cgal}

For compiling a non-shipped program, it is
recommended, to also rely on a \cmake-supported configuration using a
\texttt{CMakeLists.txt} used for configuration. 

Use the following Bourne-shell script for programs that are relatively
simple to configure:

\subsubsection{\texttt{cgal\_create\_CMakeLists\label{sec:create_cgal_CMakeLists}}}
\TTindex{cgal\_create\_CMakeLists.txt}\index{scripts!\texttt{cgal\_create\_CMakeLists}}

The Bourne-shell script \texttt{cgal\_create\_CMakeLists.txt} resides in the
\texttt{\cgalrel/scripts} directory. It can be used to create
\texttt{CMakeLists.txt} files for compiling \cgal\ applications. Executing
\texttt{cgal\_create\_CMakeLists.txt} in an application directory creates a
\texttt{CMakeLists.txt} containing rules to build the contained
application(s). Three command line options determine details of the
configuration.

\begin{description}
\item [\texttt{-s source}] If this parameter is given the script will
  create \textbf{a single executable} for 'source' linked with
  compilations of all other source files (\texttt{*.cpp}). This
  behaviour is usually needed for (graphical) demos. \\
  If the parameter is not given, the script creates \textbf{one executable for each given
  source file}.
\item [\texttt{-c com1:com2:...}] Lists components (``com1'',
  ``com2'') of \cgal\ to which the executable(s) should be linked. Valid components are \cgal's
  libraries (i.e.~``Core'', ``ImageIO'', ``Qt3'' and ``Qt4''; note
  that it only make sense to either pick ``Qt3'' or ``Qt4'') and all
  preconfigured 3rd party software, such as ``MPFI'' or ``RS3''
  ). An example is \texttt{-c Core:GMP:RS3:MPFI}

\item [\texttt{-b boost1:boost2:...}] Lists components (``boost1'',
  ``boost2'') of \boost\ to which the executable(s) should be
  linked. Valid options are, for instance, ``filesystem'' or ``program\_options''.

\end{description}

This options should suffice to create \texttt{CMakeLists.txt} script
for most directories containing programs. However, in some special
cases, it might still be required to create the script manually, for
instance, if some source files/executables need a different linking than
other source files. 

\begin{ccDeprecated}
\subsubsection{\texttt{cgal\_create\_cmake\_script\label{sec:create_cgal_cmake_script}}}
\TTindex{cgal\_create\_cmake\_script}\index{scripts!\texttt{cgal\_create\_cmake\_script}}

For backward-compatibility we still provide the 
Bourne-shell script \texttt{cgal\_create\_cmake\_script} that 
is contained in the
\texttt{\cgalrel/scripts} directory. It can be used to create
\texttt{CMakeLists.txt} files for compiling \cgal\ applications. Executing
\texttt{cgal\_create\_cmake\_script} in an application directory creates a
\texttt{CMakeLists.txt} containing rules for every \texttt{*.cpp} file
there. The script is deprecated, as it only works for applications 
with a single course file that only need libCGAL and libCGAL\_Core.
\end{ccDeprecated}

Such a shell-script simply creates a \cmake\ script. Processing it
with \cmake, searches for \cgal\ using \texttt{find\_package}. If found,
the variable \texttt{CGAL\_USE\_FILE} is set to a compilation environment \cmake\ file. Including
this file within a \cmake\ script sets up include paths and libraries to
link with \cgal\ and essential third party libraries. Beyond,
\texttt{find\_package} can demand for \texttt{COMPONENTS} of \cgal,
that is, all \cgal\ libraries libCGAL\_Core (``Core''),
libCGAL\_ImageIO (``ImageIO'') , libCGAL\_Qt3 (``Qt3'') and libCGAL\_Qt4
(``Qt4'') or optional 3rd party software such as ``MPFI'' or ``RS3''.
A user is free to create the \texttt{CMakeLists.txt}
without calling the script (manual creation).


\subsection{Custom flags in the programs using \cgal}

Normally, programs linked with \cgal\ must be compiled with the same flags
used by the compilation of \cgal\
libraries. For this reason, the \emph{very first} time
a program is configured, all the flags given by the \cmake\ variables \texttt{CMAKE\_*\_FLAGS}
are \emph{locked} in the sense that the values recorded in \texttt{CGALConfig.cmake} 
are used to override any values given by \cmake\ itself or yourself.

This does not apply to the additional flags that can be given via \texttt{CGAL\_*\_FLAGS}.

Such \emph{inherited} values are then recorded in the current \cmake\ cache for the program.
The flags are then \emph{unlocked} in the sense that at any subsequent configuration you can
provide your own flags and this time they will not be overridden.

When using the interactive \texttt{cmake-gui} the first press on \texttt{Configure} unlocks
the flags, so that you can edit them as needed. 

\begin{ccAdvanced}
The locking of flags is controlled by the variable {\tt CGAL\_DONT\_OVERRIDE\_CMAKE\_FLAGS}
which starts out FALSE and is toggled right after the flags have been loaded from
\texttt{CGALConfig.cmake}.

If you use the command line tool you can specify flags \emph{directly} by setting the
controlling variable right up front:

{\ccTexHtml{}{}
\begin{alltt}

cd \cgalrel

cmake -DCMAKE_BUILD_TYPE=Release -DCMAKE_CXX_FLAGS=-g .

cd \cgalrel/examples/Straight_skeleton_2

cmake -DCGAL_DIR=\cgalrel -DCMAKE_BUILD_TYPE=Debug -DCMAKE_CXX_FLAGS=-O2 -DCGAL_DONT_OVERRIDE_CMAKE_FLAGS=TRUE . 

\end{alltt}
}
\end{ccAdvanced}




%%
%% EOF
%%


