
\chapter{Preliminaries}
\ccChapterAuthor{CGAL Editorial Board}

\begin{ccPkgDescription}{Bounding Volumes \label{Pkg:BoundingVolumes}}
  \ccPkgHowToCiteCgal{cgal:fghhs-bv-12} 
  \ccPkgSummary{ This package
    provides algorithms for computing optimal bounding volumes of
    point sets. In d-dimensional space, the smallest enclosing sphere,
    ellipsoid (approximate), and annulus can be computed. In
    3-dimensional space, the smallest enclosing strip is available as
    well, and in 2-dimensional space, there are algorithms for a number
    of additional volumes (rectangles, parallelograms, $k=2,3,4$
    axis-aligned rectangles). The smallest enclosing
    sphere algorithm can also be applied to a set of 
    d-dimensional spheres.}
%
%\ccPkgDependsOn{}
\ccPkgIntroducedInCGAL{1.1}
\ccPkgLicense{\ccLicenseGPL}
\ccPkgDemo{2D Bounding Volumes}{bounding_volumes_2.zip}
\ccPkgIllustration{Bounding_volumes/minCircle.png}{Bounding_volumes/minCircle.png}
\end{ccPkgDescription}


This chapter lists the licenses
under which the \cgal\ datastructures and algorithms are distributed.
The chapter further explains how to control inlining, thread safety, 
code deprecation, checking of pre- and postconditions,
and how to alter the failure behavior. 

\input{Preliminaries/licenses}



\section{Marking of Special Functionality}

In this manual you will encounter sections marked as follows.

\subsection{Advanced Features}

Some functionality is considered more advanced, for example because it is
relatively low-level, or requires special care to be properly used.

\begin{ccAdvanced}
Such functionality is identified this way in the manual.
\end{ccAdvanced}

\subsection{Debugging Support Features}

Usually related to advanced features that for example may not guarantee
class invariants, some functionality is provided that helps debugging,
for example by performing invariants checks on demand.

\begin{ccDebug}
Such functionality is identified this way in the manual.
\end{ccDebug}

\subsection{Deprecated Code}

Sometimes, the \cgal\ project decides that a feature is deprecated.  This means
that it still works in the current release, but it will be removed in the next,
or a subsequent release.  This can happen when we have found a better way to do
something, and we would like to reduce the maintenance cost of \cgal\ at some
point in the future.  There is a trade-off between maintaining backward
compatibility and implementing new features more easily.

In order to help users manage the changes to apply to their code, we attempt
to make \cgal\ code emit warnings when deprecated code is used.  This can be
done using some compiler specific features.  Those warnings can be disabled
by defining the macro \ccc{CGAL_NO_DEPRECATION_WARNINGS}.  On top of this, we
also provide a macro, \ccc{CGAL_NO_DEPRECATED_CODE}, which, when defined,
disables all deprecated features.  This allows users to easily test if their
code relies on deprecated features.

\begin{ccDeprecated}
Such functionality is identified this way in the manual.
\end{ccDeprecated}


\section{Namespace CGAL}

All names introduced by \cgal, especially those documented in these
manuals, are in a namespace called \ccc{CGAL}, which is in global
scope. A user can either qualify names from \cgal\ by adding
\ccc{CGAL::}, e.g., \ccc{CGAL::Point_2< CGAL::Exact_predicates_inexact_constructions_kernel >},
make a single name from \cgal\ visible in a scope via a \ccc{using}
statement, e.g., \ccc{using CGAL::Point_2;}, and then use this name
unqualified in this scope, or even make all names from namespace
\ccc{CGAL} visible in a scope with \ccc{using namespace CGAL;}. The
latter, however, is likely to give raise to name conflicts and is
therefore not recommended.


\section{Inclusion Order of Header Files}

Not all compilers fully support standard header names. \cgal\ provides 
workarounds for these problems in \ccc{CGAL/basic.h}. Consequently, as a 
golden rule, you should always include \ccc{CGAL/basic.h} first in your 
programs (or \ccc{CGAL/Cartesian.h}, or \ccc{CGAL/Homogeneous.h}, since they 
include \ccc{CGAL/basic.h} first).


\section{Thread Safety}

\cgal\ is progressively being made thread-safe.  The guidelines which are followed
are:
\begin{itemize}
\item it should be possible to use different objects in different threads at
the same time (of the same type or not),
\item it is not safe to access the same object from different threads
at the same time, unless otherwise specified in the class documentation.
\end{itemize}

If the macro \ccc{CGAL_HAS_THREADS} is not defined, then \cgal\ assumes it can use
any thread-unsafe code (such as static variables).  By default, this macro is not
defined, unless \ccc{BOOST_HAS_THREADS} or \ccc{_OPENMP} is defined.  It is possible
to force its definition on the command line, and it is possible to prevent its default
definition by setting \ccc{CGAL_HAS_NO_THREADS} from the command line.


\section{C++11 Support}

\cgal\ is based on the \CC\ standard released in 1998 (and later refined in 2003).
A new major version of this standard has been released, and is refered to as C++11.
Some compilers and standard library implementations already provide some of the
functionality of this new standard.  For example, \gcc\ provides
a command-line switch (\ccc{-std=c++0x}  or \ccc{-std=c++11} depending on the compiler version)
which enables some of those features.

\cgal\ attempts to support this mode progressively, and already makes use of
some of these features if they are available, although no extensive support has
been implemented yet.

\section{Functor Return Types}

\cgal\ functors support the
\ccAnchor{http://www.boost.org/doc/libs/release/libs/utility/utility.htm#result_of}{result\_of}
protocol. If a functor \ccStyle{F} has the same return type across all
overloads of \ccStyle{operator()}, the nested type
\ccStyle{F::result_type} is defined to be that type. Otherwise the
return type of calling the functor with an argument of type
\ccStyle{Arg} can be accessed through
\ccStyle{boost::result_of<F(Arg)>::type}.

\input{Preliminaries/checks} % extra chapter

\section{Identifying the Version of CGAL\label{sec:cgal_version}}

\ccInclude{CGAL/config.h}

Every release of \cgal\ defines the following preprocessor macros:
\begin{description}
\item[\texttt{CGAL\_VERSION}]
     \index{CGAL_VERSION macro@{\tt CGAL\_VERSION} macro}
     -- a textual description of the current release
        (e.g., or 3.3 or 3.2.1 or 3.2.1-I-15), and 
\item[\texttt{CGAL\_VERSION\_STR}]
     \index{CGAL_VERSION_STR macro@{\tt CGAL\_VERSION\_STR} macro}
     -- same as \texttt{CGAL\_VERSION} but as a string constant token, and
\item[\texttt{CGAL\_VERSION\_NR}]
     \index{CGAL_VERSION_NR macro@{\tt CGAL\_VERSION\_NR} macro}
     -- a numerical description of the current release such that
        more recent releases have higher number.

     More precisely, it is defined as \texttt{1MMmmbiiii},
     where \texttt{MM} is the major release number (e.g. 03),
     \texttt{mm} is the minor release number (e.g. 02),
     \texttt{b} is the bug-fix release number (e.g. 0), and
     \texttt{iiii} is the internal release number (e.g. 0001). For
     public releases, the latter is defined as 1000.
     Examples: for the public release 3.2.4 this number is 
     1030241000; for internal release 3.2-I-1, it is 1030200001.
     Note that this scheme was modified around 3.2-I-30.
\item[\texttt{CGAL\_VERSION\_NUMBER(M,m,b)}]
     \index{CGAL_VERSION_NUMBER macro@{\tt CGAL\_VERSION\_NUMBER} macro}
     -- a function macro computing the version number macro
     from the M.m.b release version.  Note that the internal release
     number is dropped here.  Example: \texttt{CGAL\_VERSION\_NUMBER(3,2,4)}
     is equal to 1030241000.
\end{description}
 

\begin{ccAdvanced}
\section{Compile-time Flags to Control Inlining}
\ccIndexMainItem{code optimization}
\ccIndexMainItem{inlining}
\ccIndexMainItem{\tt inline}

Making functions inlined can, at times, improve the efficiency of your code.
However this is not always the case and it can differ for a single function
depending on the application in which it is used. Thus \cgal\ defines a set 
of compile-time macros that can be used to control whether certain functions 
are designated as inlined functions or not.  The following table lists the 
macros and their default values, which are set in one of the \cgal\ include
files.  

\begin{tabular}{l|l}
               macro name        & default \\ \hline
\ccc{CGAL_KERNEL_INLINE}         & inline \\
\ccc{CGAL_KERNEL_MEDIUM_INLINE}  &  \\
\ccc{CGAL_KERNEL_LARGE_INLINE}   &  \\
\ccc{CGAL_MEDIUM_INLINE}         & inline \\
\ccc{CGAL_LARGE_INLINE}          &  \\
\ccc{CGAL_HUGE_INLINE}           & 
\end{tabular}

If you wish to change the value of one or more of these macros,
you can simply give it a new value when compiling.  For example, to make
functions that use the macro \ccc{CGAL_KERNEL_MEDIUM_INLINE} inline functions,
you should set the value of this macro to \texttt{inline} instead of the
default blank. 

Note that setting inline manually is very fragile, especially in a template
context.  It is usually better to let the compiler select by himself which
functions should be inlined or not.
\end{ccAdvanced}
